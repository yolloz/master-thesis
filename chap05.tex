\chapter{Evaluation}

In this chapter, we evaluate the outcomes of the implementation presented in the preceding chapters. The primary goal of our work was data flow analysis of embedded code in context of Manta Flow and we need to assess whether our proposed solution achieved its intended objectives.
\par
In this evaluation, we will use an example of an AWS Glue ETL job. This example demonstrates the usage of Python Embedded Code Service for data flow analysis of embedded code in AWS Glue. It makes sense to evaluate both of these components together, because that is how they are intended to be used. We will present and explain the source of the example and then we will show and assess the resulting data lineage with respect to the predefined objectives.
\par
Note that the presented example was created to showcase the implemented functionality and might not be a real representation of an AWS Glue job. However, it does not mean that the data flow makes no sense or that actual scripts would be completely different, but rather that they would be structured differently and contain additional logic for logging etc. The example is also limited by unimplemented features because some function calls that would be used in a production code are not yet supported in Python scanner.

\section{ETL job example}






Mostly description of tests and that it works. If we are lucky we can run it on some Varo scripts and see how it works.

\section{Future work}
What can be done in the future, we can mention that Databricks is being also implemented in ECS.

%%%%%%%%%%%%%%%%%%%%%%%%%%%%%%%%%%%%%%%%%%%%%%%%%%%%%%%%%
\section{Code snippet example}

\begin{figure}[h!]
\begin{lstlisting}[language=Python] 
pass
\end{lstlisting}
\caption{Sample code for representation}
\label{fig:sample-transform-code}
\end{figure}

%%%%%%%%%%%%%%%%%%%%%%%%%%%%%%%%%%%%%%%%%%%%%%%%%%%%%%%%%%%%%%%%%%%%%%%%%%%%%%%%%%%%%%%%%%%%%%%%%%%
