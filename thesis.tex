%%% The main file. It contains definitions of basic parameters and includes all other parts.

%% Settings for single-side (simplex) printing
% Margins: left 40mm, right 25mm, top and bottom 25mm
% (but beware, LaTeX adds 1in implicitly)
\documentclass[12pt,a4paper]{report}
\setlength\textwidth{145mm}
\setlength\textheight{247mm}
\setlength\oddsidemargin{15mm}
\setlength\evensidemargin{15mm}
\setlength\topmargin{0mm}
\setlength\headsep{0mm}
\setlength\headheight{0mm}
% \openright makes the following text appear on a right-hand page
\let\openright=\clearpage

%% Settings for two-sided (duplex) printing
% \documentclass[12pt,a4paper,twoside,openright]{report}
% \setlength\textwidth{145mm}
% \setlength\textheight{247mm}
% \setlength\oddsidemargin{14.2mm}
% \setlength\evensidemargin{0mm}
% \setlength\topmargin{0mm}
% \setlength\headsep{0mm}
% \setlength\headheight{0mm}
% \let\openright=\cleardoublepage

%%%%%%%%%%%%%%%%%%%%%%%%%%%%%%%%%%%%%%%%%%%%%%%%%%%%%%%%%%%%%%%%
	\usepackage[utf8]{inputenc}	
\usepackage{tcolorbox}	
\usepackage{float}	
\newtcbox{\inlinecode}{on line, boxrule=0pt, boxsep=0pt, top=2pt, left=2pt, bottom=2pt, right=2pt, colback=gray!15, colframe=white, fontupper={\ttfamily \footnotesize}}	
   	
\usepackage[utf8]{inputenc}	
\usepackage{listings}	
\renewcommand{\lstlistingname}{Figure}% Listing -> Figure	
\renewcommand{\lstlistlistingname}{List of \lstlistingname s}% List of Listings -> List of Figures	
\usepackage{xassoccnt}	
\DeclareCoupledCounters[name=figurelistingsgroup]{figure,lstlisting}	
\makeatletter	
\AtBeginDocument{%	
  \let\c@figure\c@lstlisting	
  \let\thefigure\thelstlisting	
  \let\ftype@lstlisting\ftype@figure % give the floats the same precedence	
}	
\makeatother	
%New colors defined below	
\definecolor{codegreen}{rgb}{0,0.6,0}	
\definecolor{codegray}{rgb}{0.5,0.5,0.5}	
\definecolor{codepurple}{rgb}{0.58,0,0.82}	
\definecolor{backcolour}{rgb}{0.95,0.95,0.92}	
%Code listing style named "mystyle"	
\lstdefinestyle{mystyle}{	
  backgroundcolor=\color{backcolour},   commentstyle=\color{codegreen},	
  keywordstyle=\color{magenta},	
  numberstyle=\tiny\color{codegray},	
  stringstyle=\color{codepurple},	
  commentstyle=\ttfamily,	
  basicstyle=\ttfamily\footnotesize,	
  breakatwhitespace=false,         	
  breaklines=true,                 	
  captionpos=b,                    	
  keepspaces=true,                 	
  numbers=left,                    	
  numbersep=5pt,                  	
  showspaces=false,                	
  showstringspaces=false,	
  showtabs=false,                  	
  tabsize=2	
}	
%"mystyle" code listing set	
\lstset{style=mystyle}	

%%%%%%%%%%%%%%%%%%%%%%%%%%%%%%%%%%%%%%%%%%%%%%%%%%%%%%%%%%%%%%%%

%% Generate PDF/A-2u
\usepackage[a-2u]{pdfx}

%% Character encoding: usually latin2, cp1250 or utf8:
\usepackage[utf8]{inputenc}

%% Prefer Latin Modern fonts
\usepackage{lmodern}

%% Indent the first lines in the section/paragraph	
\usepackage{indentfirst}

%% Further useful packages (included in most LaTeX distributions)
\usepackage{amsmath}        % extensions for typesetting of math
\usepackage{amsfonts}       % math fonts
\usepackage{amsthm}         % theorems, definitions, etc.
\usepackage{bbding}         % various symbols (squares, asterisks, scissors, ...)
\usepackage{bm}             % boldface symbols (\bm)
\usepackage{graphicx}       % embedding of pictures
\usepackage{fancyvrb}       % improved verbatim environment
%\usepackage{natbib}         % citation style AUTHOR (YEAR), or AUTHOR [NUMBER]
\usepackage[nottoc]{tocbibind} % makes sure that bibliography and the lists
			    % of figures/tables are included in the table
			    % of contents
\usepackage{dcolumn}        % improved alignment of table columns
\usepackage{booktabs}       % improved horizontal lines in tables
\usepackage{paralist}       % improved enumerate and itemize
\usepackage{xcolor}         % typesetting in color
%%%%%%%%%%%%%%%%%%%%%%%%%%%%%%%%%%%%%%%%%%%%%%%%%%%%%%%%%%%%%%%%%%%%%%%%%%%%%%%
\usepackage{adjustbox}	
\usepackage{wrapfig}	
\usepackage[htt]{hyphenat}	
\usepackage[T1]{fontenc}	
\usepackage[final]{microtype}	
\usepackage{listings} %For code in appendix	
\lstset	
{ %Formatting for code in appendix	
    language=Python,	
    basicstyle=\footnotesize,	
    numbers=left,	
    stepnumber=1,	
    showstringspaces=false,	
    tabsize=2,	
    breaklines=true,	
    breakatwhitespace=false,	
    keywordstyle=\ttfamily,	
    stringstyle=\ttfamily,	
    identifierstyle=\ttfamily	
}

\usepackage[style=numeric,backend=biber,sorting=none]{biblatex}
\setcounter{biburllcpenalty}{7000}
\setcounter{biburlucpenalty}{8000}
\addbibresource{bibliography.bib}
%%%%%%%%%%%%%%%%%%%%%%%%%%%%%%%%%%%%%%%%%%%%%%%%%%%%%%%%%%%%%%%%%%%%%%%%%%%%%%%

%%% Basic information on the thesis

% Thesis title in English (exactly as in the formal assignment)
\def\ThesisTitle{Data Lineage Analysis Service for Embedded Code}

% Author of the thesis
\def\ThesisAuthor{Michal Jurčo}

% Year when the thesis is submitted
\def\YearSubmitted{2022}

% Name of the department or institute, where the work was officially assigned
% (according to the Organizational Structure of MFF UK in English,
% or a full name of a department outside MFF)
\def\Department{Department of Distributed and Dependable Systems}

% Is it a department (katedra), or an institute (ústav)?
\def\DeptType{Department}

% Thesis supervisor: name, surname and titles
\def\Supervisor{doc. RNDr. Pavel Parízek, Ph.D.}

% Supervisor's department (again according to Organizational structure of MFF)
\def\SupervisorsDepartment{Department of Distributed and Dependable Systems}

% Study programme and specialization
\def\StudyProgramme{Computer Science}
\def\StudyBranch{Software and Data Engineering}

% An optional dedication: you can thank whomever you wish (your supervisor,
% consultant, a person who lent the software, etc.)
\def\Dedication{%
Dedication.
}

% Abstract (recommended length around 80-200 words; this is not a copy of your thesis assignment!)
\def\Abstract{%
Abstract.
}

% 3 to 5 keywords (recommended), each enclosed in curly braces
\def\Keywords{%
{key} {words}
}

%% The hyperref package for clickable links in PDF and also for storing
%% metadata to PDF (including the table of contents).
%% Most settings are pre-set by the pdfx package.
\hypersetup{unicode}
\hypersetup{breaklinks=true}
\hypersetup{hidelinks}

% Definitions of macros (see description inside)
\include{macros}

% Title page and various mandatory informational pages
\begin{document}
\include{title}

%%% A page with automatically generated table of contents of the master thesis

\tableofcontents

%%% Each chapter is kept in a separate file
\chapter{Introduction}

Prior to the advent of computers, data processing was predominantly a manual task that involved a considerable amount of time, effort, and the potential for human errors. Organizations, particularly businesses, faced challenges in handling large volumes of data efficiently and accurately. This led to a demand for automated systems that could streamline data processing and eliminate the limitations associated with manual methods. The development of computers offered a solution to these challenges by providing a means to automate data-related tasks. It allowed storing and processing larger amounts of data than ever before.
\par
A couple of decades later, data is one of the most important resources for many companies. It helps them run their business more efficiently and make better business decisions. Data pipelines have emerged as crucial components in modern business infrastructures. In general, a data pipeline refers to a system or framework that facilitates the flow of data from various sources to its destination, typically for processing, analysis, storage, or visualization. It is a series of interconnected steps or processes that enable the extraction, transformation, and loading (ETL) of data, ensuring that it moves efficiently and reliably through the pipeline.
\par
A simple example of a data pipeline might be saving contents of a form filled by a customer on a web page to a database. Another, a more complex one might be a preparation of marketing success report where data from recent sales stored in an accounting system are correlated with a list of latest advertising campaign exported from a marketing tool and compared with customer satisfaction form result which might be stored in a database as in the previous example. It is easy to see that these pipelines might become long and complex and that they can be chained one after the other. Less obvious is that they are often facilitated by multiple systems. It would be easier to manage if they were facilitated by one, but there are often reasons why that is not possible. There is no universal tool, each serves a different purpose. A database is great for storing data and fast queries over large data sets, ETL tools are good in data transformations, reporting tools are great for data visualization and analytic and machine learning tools are essential for data science. There might even be a legacy system that cannot be easily migrated to a different platform. They all compose a data environment which serves one purpose - to run business more effectively and make good business decisions.
\par
As the Greek philosopher Heraclitus said, \textit{the only constant in life is change}. This statement is fitting for data environments, because they always change. A new column is added to a schema, two are removed, a new process is introduced or an existing one needs to be modified or fixed etc. A change may span across multiple systems or a modification in one system may influence other systems that depend on it. Planning it can become a nightmare, because even if there is a support for impact analysis in each system, there is none environment-wide and has to be prepared manually. That is why additional processes were developed that help making changes with predictable outputs, without causing errors and that ensure data correctness afterwards. One of such processes is called \textit{data lineage}. 
%% TODO: add also a note about data governance/data quality and migrations

\section{Data lineage}

Data lineage is a process of mapping and visualizing data flows within a data environment. It tracks data as they flow from various sources to their destinations and their transformations with aim to help manage and develop data environments. It provides a comprehensive understanding of how data is acquired, manipulated, and utilized within an organization. Data lineage offers several benefits to businesses and data professionals. Firstly, it enhances data governance and regulatory compliance by ensuring transparency and traceability of data. It enables organizations to meet the requirements of various regulations such as GDPR or CCPA. Secondly, data lineage improves data quality and accuracy by identifying data inconsistencies, errors, or gaps in the data flow. This helps in identifying and rectifying issues promptly, leading to reliable and trustworthy data insights. Additionally, data lineage facilitates data discovery, data integration, and data analytics processes, as it provides a clear understanding of data origins and transformations. It enables faster troubleshooting and root cause analysis, reducing the time and effort required for resolving data-related issues. Overall, data lineage plays a crucial role in maximizing the value of data assets and ensuring data integrity, trust, and accountability within an organization.
\par
Data lineage can be obtained by hand from teams of analysts that map data environments or, more recently, using one of the automated systems that are being developed. Manual data lineage analysis is time- and labor-intensive, so developing automated solutions can provide more up-to-date results and decrease costs.
\par
\textit{Manta Flow} is an automated data lineage platform. It can analyze complex data environments consisting of various databases, data integration and reporting tools and applications in Java, C\# or Python. Using metadata extracted from each connection to a system, Manta Flow computes data lineage graphs where vertices represent data sources and directed edges represent data flows between them. At first, a graph is constructed and stored for each individual connection. When the data lineage is visualized, the graphs are retrieved from the repository and combined together to show a graph of the entire environment.

\begin{figure}[ht]\centering
\includegraphics[width=1.0\textwidth]{img/graph_example.png}
\caption{An example of combined data lineage graph}
\label{figGraphExample}
\end{figure}  

\par
To illustrate this with an example, let us have an ETL tool which contains a pipeline writing data into table A of database XYZ and a reporting tool which visualizes data from this table in a report. The analysis would produce three graphs:
\begin{itemize}
    \item an ETL tool graph which contains data pipeline data flow ending with a write to table A,
    \item a database XYZ graph which represents table A and its columns,
    \item a reporting tool graph that contains data read from table A into a report. 
\end{itemize}
These graphs would be visualized as one unified data lineage, as seen in figure~\ref{figGraphExample}.

\section{Embedded code}
Various data processing, management and analytic tools have been developed to allow their users to extract important pieces of information from the vast amounts of data they collect. These tools often provide graphical interface for creating data pipelines consisting of commonly used data sources and transformations. Using such tool opens data pipeline management to a wider, less technically proficient audience, so business-oriented employees can be involved more closely in the development. An example of such tool may be AWS Glue platform where users may define their ETL pipeline using graphical interface with multiple data sources, transformations and joins without having to write a single line of code. However, data can be very dynamic and different and it would be difficult to define every possible data transformation that the users might require. In such cases these tools often allow extending the pipeline with embedded code.
\par
Embedded code is a piece of code provided by a user that is safely executed in the context of the tool and can perform (almost) any desired task. It is often a function or a script. Popular examples of tools that support embedded code include already-mentioned AWS Glue, but also Databricks platform, Snowflake data cloud or SQL Server Integration Services (SSIS) etc. Programming languages used in embedded code often include the popular and well-known ones such as Python, Java, Scala or C\#.
\par
Embedded code can be used to perform a data transformation in an ETL pipeline that is not included in the toolbox, read data from an unsupported data source or to create a user-defined database function that cannot be written efficiently in SQL.
\par
Data flow analysis of embedded code is a crucial missing link in Manta Flow. It can analyze both data pipeline metadata and standalone Python, Java or C\# applications, but there is no support when such a piece of code is a part of the pipeline. Missing embedded code data lineage causes logical gaps in the holistic data lineage and decreases its usability, because these gaps have to be filled in by hand. Following the recent surge in demand for data science and machine learning solutions where especially Python is the programming language of choice, the number of enterprise data environments which contain embedded code has also risen significantly. This drives the market demand for data lineage solutions that can cope with it. As there are currently no solutions that can reliably provide it, extending current capabilities of Manta Flow in data flow analysis of applications to embedded code shall provide it a significant competitive advantage.

\section{AWS Glue}

AWS Glue is a serverless data integration service that makes it easier to discover, prepare, move, and integrate data from multiple sources for analytics, machine learning and application development. It performs data processing on Apache Spark engine in a cloud environment. Data pipelines are defined using embedded code, supported programming languages are Python and Scala. They can also be created from the GUI, where the tool generates the corresponding pipeline code. This service is especially convenient for companies that already use other AWS services as it can efficiently use such resources.
\par
AWS Glue has been on top of the list of technologies to be supported by Manta Flow based on customer enquiries. There is currently very limited support for automated AWS Glue data lineage on the market, so having it provides a competitive advantage. As the pipelines consist of embedded code, analyzing it is the best, although a very difficult way to extract data lineage information. However, Manta Flow can already analyze Python and Bytecode (Scala is compiled into Bytecode) applications. Extending the analysis support with embedded code is therefore a direct prerequisite for a successful AWS Glue data lineage analysis.

\section{Goals}

The goal of this thesis is to design a data lineage analysis service for embedded code in Manta Flow that will enable integration of data lineage graph from data processing and analytic tools with the data lineage graph derived from the embedded code.
\par
One of the main tasks is to create a solid design of the service that should be easily extendable with support for new tools and their embedded code in the future. Benefits and usefulness of this design will be then demonstrated on a prototype implementation of the service for AWS Glue and the embedded code written in Python.
\par
Other specific tasks include a proof-of-concept implementation of a metadata extractor for AWS Glue and modifications to the existing Python scanner. A very important aspect of the service is high performance, because it will be called many times during a run of the Manta Flow analysis platform, specifically every time the analysis processes a statement that executes a piece of embedded code.

\section{Glossary}

Let us define a few important terms that are often used in this work.
%% TODO
\begin{itemize}
    \item Data lineage
    \item Data flow
    \item Manta Flow
    \item Manta scanner
    \item We will use the term \textit{data technology} to uniformly reference databases, ETL and reporting tools. This term shall simplify naming all systems, frameworks, platforms and tools that can be used for data processing and management in the work.
    \item Embedded code
    \item Metadata
    \item In the context of embedded code analysis, the \textbf{source technology} is the data technology that uses embedded code, or from a different point of view, a data technology that is the source of embedded code.
\end{itemize}

\section{Outline}

This thesis is split into several chapters. In introduction we briefly describe the motivation and the goal of the thesis. In the second chapter we describe important aspects of MANTA Flow platform with which the service developed in this thesis is integrated. Chapter 3 is dedicated to a detailed problem analysis and we formulate our requirements there. Chapter 4 delves into design and implementation of embedded code service and changes that were made to Python scanner. In chapter 5 we take a look at the design and proof-of-concept implementation of AWS Glue scanner, which uses embedded code service for data flow analysis of embedded Python code. In chapter 6 we demonstrate the functionality of implemented features on several examples and discuss the limitations of the implementation. Finally, in conclusion we sum up what we achieved in this work and how we did it.
\chapter{MANTA Flow platform}

\section{MANTA Flow description}
Description of MANTA Flow, description of scanners etc., explaining terminology
\section{Language scanners}
Description of language scanners, their composition, how they work
\section{Query service}
Description of query service, how it works, similarities to what we are trying to achieve.

\chapter{Requirements and Analysis}

\section{Problem overview}
An extended problem overview, analysis of supported/desired technologies that support embedded code.

\section{Requirements}
Functional and qualitative requirements

\section{AWS Glue analysis}
Analysis of AWS Glue service.

\section{Python scanner analysis}
Analysis of Python scanner w.r.t. Embedded Code service

\chapter{Design And Implemetation Of Embedded Code Service}

Based on the analysis conducted in the previous chapter we now have a clear understanding what Embedded Code Service is, what is its purpose and how it is going to solve outlined problems. In this chapter, we are going to introduce and reason about its design. We are also going to explain what changes need to be done to programming language scanners in order for them to be integrated with Embedded Code Service. At the end of the chapter we will present interesting parts of the implementation based on this design.

\section{Embedded Code Service design}

Firstly, let us summarize the steps that Embedded Code Service has to execute, because it might not be clearly obvious from previous chapters. The goal of the service is to perform data flow analysis of embedded code using one of the already existing scanners to create a data lineage graph of that code which will then be merged with the graph produced by source technology scanner. A programming language scanner works in three steps: extraction of the input, data flow analysis and generating Manta graph. Embedded Code Service needs to perform input orchestration using the provided configuration, then launch all three stages of a scanner and after that help with merging the graphs. The workflow looks as follows (active components are written in parentheses):
\begin{enumerate}
    \item Input orchestration (Embedded Code Service)
    \item Input extraction (scanner's Extractor)
    \item Data flow analysis (scanner's Reader)
    \item Generating lineage graph (Intermediate Dataflow Generator)
    \item Merging graphs (source technology scanner with the help of Embedded Code Service)
\end{enumerate}

\subsection{Multiple programming languages}

The first decision we have to make is whether we want to implement one universal service that can analyze embedded code written in any programming language or whether we want to have specific implementations for each programming language. This decision will greatly influence how the interface of the service is designed. Comparing the two approaches, we chose specific implementations as a more suitable solution.
\par
An initial idea seems to be a universal service, because that is how Dataflow Query Service is implemented. A common service promotes code reuse as multiple parts of the problem will be solved in a similar way regardless of used programming language and data technology, such as merging the graphs of embedded code and source technology. We can also find similarities between the data technologies (e.g., stored procedures written in embedded code in databases) regardless of the programming language used. One service also means that only one component will need to be maintained. The downside of having one service is that its interface needs to be universal, so if one programming language has a different requirement or requires a specific modification, these changes will have to be reflected for other programming languages as well.
\par
One could argue that another benefit of one service is that we have access to the analysis of any embedded code we may find, but that turns out not to be as beneficial as it may sound. In reality, the language of embedded code is always know, so it is easy to use a specific service to analyze it. Having a universal service has been crucial in case of Dataflow Query Service, because it supports recognition of the SQL dialect used in the query, but in case of Embedded Code Service such feature is not needed. When implementing specific services, we can still reach similar code reuse by grouping common logic in base classes, which makes one less argument in favor of a common service. An important advantage that multiple services provide is that they not only allow the interfaces to be tailored to the needs of the programming language scanner, but they also allow different development pace for each service. The fact is that a service for Python is much more preferred by the stakeholders because of its potential and thus a lot more resources are dedicated to working on it. The last argument carries great significance due to its profound impact on the development process, so we decided to implement multiple services, each dedicated for one programming language.
\par
Figure~\ref{fig:ECSbasedesign} shows how a specific service processes embedded code as well as components that is utilizes. The diagram is generic for any programming language.

\begin{figure}[ht]\centering
\includegraphics[width=1.0\textwidth]{img/Embedded code service base design.png}
\caption{Diagram of Embedded Code Service workflow}
\label{fig:ECSbasedesign}
\end{figure}   

\subsection{Orchestration}

All contemporary programming languages use some form of import mechanism where the developers can import additional libraries and frameworks. We can see that in embedded code too, often there is a mechanism of adding external libraries to be imported and used by the code. For example in AWS Glue it is possible to define a job argument with a path of Amazon S3 object which contains a custom Python library. When AWS Glue executes a Python job, firstly it checks the S3 location and when it conforms to the required format, it copies the files to the internal working directory next to the job script. When a job is started, Python's import mechanism is able to find these additional libraries and successfully use them in the job script.
\par
Additionally, some technologies (e.g. SAS, Databricks) perform additional orchestration to run the embedded code correctly, such as injecting it in a well-defined class, so this envelope does not have to be written by the user every time, adding the desired imports and only then the embedded code is run on the execution engine used by that technology. This process has to be mimicked by Embedded Code Service before analyzing the code by the scanner to provide an equivalent of the actual code that is executed.
\par
The orchestration process would generally be different for each technology and programming language, but there are some common steps and some repeating patterns. This needs to be reflected in the design of orchestration so the part that needs to be implemented anew when support for new source technology is added is clearly distinguished from common parts. We can use \textit{template method} design pattern, which implements common parts of the process in the base class and lets subclasses redefine certain parts of it. That way we clearly define what can and needs to be implemented.

\subsection{Result}

When the analysis of embedded code is finished, the service shall return a result to source technology scanner. The result is a graph that needs to be merged with another graph and may contain pin nodes that need to be connected. Rather than returning the unfinished graph, we shall return an object that wraps it and provides an interface for connecting pin nodes and merging. The justification is simple, this graph is not always in a valid state so it should be hidden from plain view and unwanted modifications.

\subsubsection{Pin node mapping}
First thing to be done with the result is pin node mapping. Pin nodes can be found in the graph by enumerating its nodes and filtering those with pin node type. Each pin node can be distinguished from others by its name.
\par
The source technology scanner performing the mapping should receive the information about created pin nodes from an \textit{Insight}. A mapping simply contains a pin node and a node to be mapped to. We might want to connect pin nodes by iterating over them and for each one we create a mapping or we might want to iterate over records in an \textit{Insight}, retrieve the pin node by its name and then create the mapping. Each approach is better in different situations, so the result shall support both operations.
\par
When a mapping is created, it is important to specify whether the pin node is an input node or an output node (relatively to embedded code). The direction decides the orientation of the edge when a pin node is merged to the source technology graph.
\par
During the mapping process, the mapping information is stored and merging is deferred until the end so that the graph remains unchanged throughout the process.

\subsubsection{Merging}
After all pin nodes are mapped, a merge operation can start. This operation can be done only once, because it changes the source technology graph. If it was done multiple times, the graph could be damaged.
\begin{figure}[ht]\centering
\includegraphics[width=1.0\textwidth]{img/contraction.png}
\caption{The process of merging and contracting a pin node}
\label{fig:contraction}
\end{figure}   
\par
The merging process starts by adding the pin node into the source technology graph along with an edge that connects it to the mapped node. After that, pin node needs to be contracted and removed. That is done by first enumerating its incoming and outgoing edges and adding new edges starting at the starting points of incoming edges and ending at the end points of outgoing edges. After that, pin node and all edges associated with it are removed. A process of contracting a node can be seen in Figure~\ref{fig:contraction}.
\par
When an edge is added to a graph, both its nodes are also added if they were not already present in the graph. After all pin nodes are added to the source technology graph and contracted, all remaining nodes and edges are added and the merging is complete. 





%%-subsection-%% 
\subsection{Running the scanner for the embedded code without Manta Flow hooks}
Similarly to the previous problem, this also involves the interface of programming language scanners. These are currently designed to be started by a Manta Flow scenario, responding to hooks. It is desired to modify this interface to enable starting the scanners from Embedded Code Service. These modifications should not be big (the scanners are already quite well designed), but it is important to keep in mind that they need to be finished before a language scanner can be added to the Embedded Code Service. 






%%----- SECTION -----%% Analysis of Python scanner w.r.t. Embedded Code service
\section{Python scanner analysis}

Since the Python scanner wasn't initially designed for running embedded code, we need to analyze what are the required changes in order to support Embedded Code Service (further referenced to as ECS).

\subsection{Interface and Spring configuration}
Python Scanner is designed to analyze user inputs. The source code is provided by users and its location together with other settings is provided in scenario configuration. ECS also receives the input, but not the rest of the configuration. This configuration is supposed to be built during its runtime in input orchestration phase of processing embedded code (see DEV-24356: Python embedded code service - Python scanner interface modifications
DONE
). 
\par
Upon analyzing the current state of the components it is clear that they were designed as single-purpose components - they are constructed with all functional elements and task configuration they require. Therefore, the lifecycle of such component ends when it finishes its task as new component has to be constructed for a new task. This is feasible for language scanners running from CLI, because each scenario executes each task once. With ECS the expected lifecycle of components is different. The service is constructed once but can be used for multiple tasks (e.g., analysis of multiple scripts that are a part of one ETL pipeline).
\par
There are two ways how we can modify current components to support both approaches efficiently. Either add a factory for a component that stores the functional elements for its construction and allow passing the configuration as a parameter in a factory method which constructs the component. This approach is suitable for components which hold a complex internal state and it would be costly to refactor it to a stateless component. Also, it is a cheap modification, there are no changes in code, only a factory needs to be added and this change needs to be reflected in Spring configuration.
\par
The other approach is to modify the component to become a stateless component and the task configuration can be passed using dependency injection. This approach is suitable for components that have a single entry-point - a single method that is invoked to complete the task. In such cases the internal state can be kept internally during the execution of the method and thrown away at the end. Alternatively, this state can be passed between the invocations of multiple methods of the component using dependency injection.

\subsection{CLI}
We need to consider one more thing. The components were designed to work with CLI and their interface is customized for that purpose. ECS doesn’t need to implement such interface. Often this interface doesn’t provide exactly the functionality we need and performs additional tasks that are not required (e.g., serializing output to disk). It would be useful to decouple implementation of the functional component from the CLI interface so the functional component can be used by both CLI and ECS but it can have its own interface to perform required tasks.

\subsection{Components}
There are three major components: Extractor, Reader and Dataflow Generator. These are the components that need to be instantiated for both CLI scanner and Embedded Code Service. We therefore need to focus on their interfaces in the scope of this task.

\subsubsection{Extractor}
Extractor almost satisfies our conditions. The configuration is passed to Extractor in constructor, but it is not necessarily needed. It contains some file paths that are read-only and need to be available during extraction. This component can be easily refactored using the first approach. For CLI use-case the configuration is already stored in ExtractorTask that stores both configuration and extractor, so no Spring configuration changes are required.

\subsubsection{Reader}
Reader is a highly state-dependent component. It would require some effort to remove all application-specific information. However, the current implementation requires some refactoring as it does more things than it should. Also, Reader is currently one component that implements CLI interface and performs the tasks. To satisfy all our conditions we need to:
\begin{enumerate}
    \item Decouple CLI interface from the Reader implementation
    \item Identify static sub-components of Reader
    \item Identify task-specific sub-components of Reader
\end{enumerate}
After the refactoring, static sub-components are a part of EntryPointAnalyzer component which can perform the analysis of a single entry point. Task-specific sub-components are a part of AnalyzerConfiguration. This configuration is like a small toolbox that the analyzer uses during the analysis and is provided for each entry point analysis. Lastly, PythonReader component is the implementation of CLI interface and internally uses an EntryPointAnalyzer and its own static AnalyzerConfiguration for its task. 
\par
With these modifications, the Spring configuration needs to be split in two parts. One is general configuration that provides initialized reusable EntryPointAnalyzer. The other is CLI specific that prepares the AnalyzerConfiguration and binds it with PythonReader.

\subsubsection{Dataflow Generator}
Dataflow Generator was reworked during this analysis so we’ll reference only the current implementation after the rework. It was much more simplified and its dependencies were reduced which makes it much easier to use in various contexts. Currently it requires program ID at construction time which is only used during dataflow generation, so it can be easily refactored to be supplied at that time instead and therefore become stateless. It also requires CLI interface decoupling.
\par
After the refactoring, a new DataflowGenerator class is the stateless component of Dataflow Generator. To transform a graph it needs ProgramConfiguration which holds only the program ID, but it is convenient to be a class so it can be instantiated as a bean. Lastly, the TransformationTask is now just the implementation of the CLI interface, previously it represented all of these new components.

\subsubsection{Configuration}
The Spring configuration of Intermediate Dataflow Generator is layered into multiple levels to introduce modularity. This modularity is important as there are currently 3 language scanners that utilize the generator and each of them is slightly different. Moreover, with the introduction of Embedded Code Service, there is another use-case with different needs.  This design reduces the amount of beans that it depends on to a bare minimum and avoids the need to define beans that are unused.
\par
Currently, there are two possible configuration compositions. One is for scanners that are not supported in Embedded Code Service and the other for those that are. For that purpose the configuration of common beans is split in two where the base context configuration sets up the core component of Dataflow Generator that can be used by the service and the context sets up the dataflow task bean that is required for CLI integration. These configurations can then be included in platform specific configuration that provides platform-specific beans.

\subsubsection{Composition without Embedded Code Service}
Without Embedded Code Service the situation is simple as there is only one use-case, so only the platform-specific beans need to be provided.
\begin{figure}[ht]\centering
\includegraphics[width=1.0\textwidth]{img/Intermediate Dataflow Generator Configuration-scanner-only composition.png}
\caption{Composition of configurations as they are included in each other}
\label{fig01:ECSbasedesign03}
\end{figure}   
\begin{figure}[ht]\centering
\includegraphics[width=1.0\textwidth]{img/Intermediate Dataflow Generator Configuration-scanner-only dependency.png}
\caption{Dependencies of configurations and their beans}
\label{fig01:ECSbasedesign04}
\end{figure}   

\subsubsection{Composition with Embedded Code Service}
When Embedded Code Service is involved, there needs to be one configuration that satisfies the CLI interfaces and there can be another one that satisfies Code Service needs. Embedded Code Service does not need the CLI interfaces, in fact, they add useless functionality for such use-case, so its best to avoid id. Moreover, in this use-case there could be multiple program configurations passed to Dataflow Generator during its lifecycle so it cannot be defined statically in Spring. To satisfy these needs the platform configurations is split in two parts. First part is configuration of common beans for both use-cases and then there are two specializations that utilize common beans - scanner (CLI) and service specialization.
\begin{figure}[ht]\centering
\includegraphics[width=1.0\textwidth]{img/Intermediate Dataflow Generator Configuration-code service-aware composition.png}
\caption{Composition of configurations as they are included in each other}
\label{fig01:ECSbasedesign05}
\end{figure}  
\begin{figure}[ht]\centering
\includegraphics[width=1.0\textwidth]{img/Intermediate Dataflow Generator Configuration-code service-aware dependency.png}
\caption{Dependencies of configurations and their beans}
\label{fig01:ECSbasedesign06}
\end{figure} 





\chapter{AWS Glue Scanner}

The previous chapters were devoted to Embedded Code Service. In this chapter, we will cover the details of the development of AWS Glue scanner prototype. AWS Glue is one of the data technologies that uses embedded code, so we will see how Embedded Code Service service can be used in data lineage analysis. To begin, we will describe AWS Glue and discuss it in terms of data lineage analysis. Next, we discuss and justify the design of a minimum viable product (MVP) of AWS Glue  scanner. Finally, we will look at its implementation. 

%%----- SECTION -----%%
\section{Motivation}

We have already briefly introduced AWS Glue in Section~\ref{sec:sourceTechnologyAnalysis}. Before we dive deeper into its analysis, we shall explain why we picked AWS Glue to demonstrate Embedded Code Service capabilities.
\par
The decision to implement Embedded Code Service in Manta Flow was motivated by the demand to analyze embedded code in several data technologies. Out of the existing programming language scanners in Manta Flow, Python scanner seemed to be the most prospective one to offer attractive value to customers for two reasons: it is widely used and the scanner yielded promising data lineage results.
\par
At this point, there were two options for selecting the data technology for MVP implementation. We could either choose an already supported one that uses embedded Python code to extend its capabilities, or create a new scanner for an unsupported data technology. For the first option, we could choose SAS, but there was no demand for such feature as Python support has only recently been added into it. For the second option, there have long been plans for AWS Glue and Databricks scanners based on customer demand, both of which use Python extensively. Out of these two, AWS Glue uses Python in a more straight-forward way as AWS Glue jobs are in fact complete Python scripts, which was close to what Python scanner could already analyze with limited results.

\section{AWS Glue analysis}

Before we can start designing the scanner, we need to analyze how AWS Glue works, how it processes data, what metadata it stores and how we can read it etc. 

\subsection{Overview}

AWS Glue is a cloud-based data integration service for discovering, cataloging and transforming data. It is a \textit{serverless} service, which means that there is no dedicated server which requires setup or maintenance. Each execution is managed by AWS Glue which allocates computing capacity on the machines present in a data center, executes the task and then frees the resources for any other task. The customer does not need to perform any maintenance, they only pay for used computing capacity and instead may focus on developing their data processes.
\par
AWS Glue provides its users with several key features:
\begin{enumerate}
    \item \textbf{Data Catalog}: AWS Glue includes a centralized metadata repository, known as Data Catalog. It stores metadata information about the data sources, tables, and schemas, making it easier to discover and understand the available data.
    \item \textbf{ETL jobs}: AWS Glue allows users to define and run ETL jobs using a visual interface or by writing custom code. ETL jobs enable data transformations such as filtering, aggregating, and joining data from different sources.
    \item \textbf{Data crawling}: AWS Glue can automatically discover and catalog data from various sources, including databases, data lakes, and file systems. It uses crawlers to scan the data sources, infer schemas, and create tables in the Data Catalog.
    \item \textbf{Data preparation}: AWS Glue provides capabilities for cleaning and preparing data before it is used for analysis. It offers built-in transformations and mappings to standardize and transform data, as well as options for creating custom transformations using Python or Scala.
    \item \textbf{Integration with other AWS services}: AWS Glue integrates with other AWS services, such as Amazon S3, Amazon Redshift, and Amazon Athena, allowing users to seamlessly move and transform data between these AWS services.
\end{enumerate}
\par
These features are split into two modules: \textit{Data Integration and ETL} and \textit{Data Catalog}.

\subsubsection{Data Integration and ETL}
The core of data integration in AWS Glue are ETL jobs executed on Apache Spark. Apache Spark is an open-source, distributed computing system that enables fast and flexible processing and analysis of large-scale data sets. It utilizes in-memory computing to accelerate iterative computations, making it ideal for big data workloads. With its distributed architecture, Spark can seamlessly distribute data and processing tasks across a cluster of computers, enabling parallel execution and efficient resource utilization. Spark provides a rich set of libraries and APIs for various data processing tasks available in several programming languages including Python and Scala.
\par
AWS Glue ETL jobs always contain a job script written in Python or Scala which is executed on Spark cluster configured in AWS Glue. Besides writing your own code, AWS Glue provides a tool with graphical user interface for creating ETL jobs. They can be created using transformations and data sources from the tool's toolbox. The tool auto-generates Python script based on the visualization. This script code can then be authored, but after that it can no longer be modified in the graphical tool.
We can see an example of an ETL job created in this tool in Figure~\ref{fig:visualJob}. This job reads a table from Data Catalog, renames some of the columns and filters out others and stores the result in a JSON file on Amazon S3, creating a Catalog table for it. Figure~\ref{fig:visualScript} shows the code generated by the tool and illustrates how ETL jobs written in Python look like.
\par
Another part of data integration is a scheduler for running jobs periodically or on custom triggers. The jobs can also be organized in a \textit{Workflow}, which is a different form of scheduling where multiple jobs can be executed in the order defined in the Workflow including various conditions and triggers.

\begin{figure}[ht]\centering
\includegraphics[height=0.6\textwidth]{img/job_example_visualization.png}
\caption{An example of an ETL job created in the graphical tool of AWS Glue}
\label{fig:visualJob}
\end{figure}

\begin{lstlisting}[language=Python,caption=Code script generated in AWS Glue for the ETL job created in a visual tool shown in Figure~\ref{fig:visualJob},label=fig:visualScript]
import sys
from awsglue.transforms import *
from awsglue.utils import getResolvedOptions
from pyspark.context import SparkContext
from awsglue.context import GlueContext
from awsglue.job import Job

args = getResolvedOptions(sys.argv, ["JOB_NAME"])
sc = SparkContext()
glueContext = GlueContext(sc)
spark = glueContext.spark_session
job = Job(glueContext)
job.init(args["JOB_NAME"], args)

# Script generated for node S3 bucket
S3bucket_node1 = glueContext.create_dynamic_frame.from_catalog(
    database="example_db", table_name="wdicountry_csv", transformation_ctx="S3bucket_node1"
)

# Script generated for node ApplyMapping
ApplyMapping_node2 = ApplyMapping.apply(
    frame=S3bucket_node1,
    mappings=[
        ("country code", "string", "country code", "string"),
        ("short name", "string", "short name", "string"),
        ("table name", "string", "table name", "string"),
        ("long name", "string", "full name", "string"),
        ("2-alpha code", "string", "2-alpha code", "string"),
        ("currency unit", "string", "currency", "string"),
    ],
    transformation_ctx="ApplyMapping_node2",
)

# Script generated for node S3 bucket
S3bucket_node3 = glueContext.getSink(
    path="s3://examplebucket/wdi_country_filtered/",
    connection_type="s3",
    updateBehavior="UPDATE_IN_DATABASE",
    partitionKeys=[],
    enableUpdateCatalog=True,
    transformation_ctx="S3bucket_node3",
)
S3bucket_node3.setCatalogInfo(
    catalogDatabase="example_db", catalogTableName="wdi_country_filtered"
)
S3bucket_node3.setFormat("json")
S3bucket_node3.writeFrame(ApplyMapping_node2)
job.commit()
\end{lstlisting}

\subsubsection{Data Catalog}
Data catalog is a centralized metadata repository than can not only be used in AWS Glue, but also in other AWS services. The purpose of Data Catalog is to organize many different data sources in a comprehensible catalog which improves data discovery and utilization. In big data environments and especially in cloud, there are many different data sources with different access rules, structure, format and schema. Data Catalog extracts this metadata from each of the data sources and stores it in an abstract structure consisting of databases, tables and columns. A user of Data Catalog then has the ability to uniformly browse and examine them regardless of the actual data source type. Data Catalog can also be used in ETL jobs to simplify reading or writing data, because each resource from Data Catalog is treated the same way in code.
\par
New resources can be added to Data Catalog manually, but its core feature is automated data \textit{crawling} and \textit{classification}. Crawling is the process of exploring resources stored in a data source and classification is the process of inferring data schema in each of the resources. A common use case is to periodically crawl a bucket in Amazon S3 to discover new files and add their schema to Data Catalog using a classifier. AWS Glue provides crawlers and classifiers for most of the common data sources and formats, but it is also possible to write custom ones or use the community-provided ones from AWS marketplace.

\subsection{Data lineage in AWS Glue}
To correctly analyze data lineage in AWS Glue, we must first explore which parts of it participate in data pipelines and may contain data flows. It is easy to see that we must analyze ETL jobs as they contain ETL pipeline code, and Data Catalog which contains metadata of data sources that could be used in ETL jobs. It turns out that this is all that we need for complete data lineage.
\par
While Workflows seem like they could participate in data lineage, in fact they are used just for job scheduling. If jobs in a Workflow depend on each other, we can see it from just analyzing the jobs, because they would use common data sources through which they would be connected in the data lineage graph. The order in which they are executed does not play any role in data lineage analysis.
\par
Crawlers and classifiers in Data Catalog may also contain executable code, but this code just enumerates resources present in a data source in case of crawlers and infers data schema of these resources in case of classifiers. There are no data flows in the sense of data lineage analysis, no values are moving from one place to another. They do not need to be analyzed, but we can read the results of their work in Data Catalog.

\subsection{AWS Glue API}
Now that we know which entities of AWS Glue we want to analyze, let us have a look at how we can get access to their metadata which we will need for the analysis. AWS Glue is usually managed from AWS console, which is a web application for controlling any AWS service. For programmatic access AWS services also provide rich API. It is common that any action that can be made in AWS console has an equivalent support in API. AWS is one of the biggest providers of cloud services worldwide so it should not be a surprise that there are many available SDKs (software development kits) for popular programming languages that support using APIs of AWS services. These SDKs are released under Apache License so we are free to use and distribute them and we can even modify them should we need to do so.
\par
As Manta Flow scanners are developed in Java, naturally the first option was to explore SDK for Java. AWS Java SDK is currently in version \textit{2.x}. This SDK contains multiple packages for each AWS service including AWS Glue. There is a comprehensible API reference and documentation for this SDK available on AWS Glue website as well as a library of examples available on GitHub which explains common usage of the SDK. Overall, this SDK is suitable to fulfill our requirements and needs. We can use it to extract all necessary metadata from AWS Glue.

\subsubsection{Security and access permissions}
It is important to understand how security and access permissions work in AWS Glue. The scanner has to read metadata of sensitive resources. It is important that it can do so in a secure way, otherwise Manta Flow users will not be willing to provide required permissions. Additionally, we need to be able to explain the users exactly what credentials and permissions we need to set up a connection for AWS Glue.
\par
AWS provides a web service called AWS IAM (AWS Identity and Access Management) that helps managing access to AWS resources securely. IAM allows controlling and managing user identities, permissions, and access to various AWS services and resources within an organization.
\par
With IAM, one can create and manage IAM users, groups, and roles. IAM users are specific identities that represent individuals or applications that interact with AWS services. IAM groups are collections of users with similar permissions, making it easier to manage permissions for multiple users. IAM roles are used to delegate access permissions to entities outside of organisation's AWS account, such as other AWS accounts or services.
\par
IAM enables setting fine-grained permissions for users, allowing to control which actions they can perform and which resources they can access. It follows the principle of least privilege, which means users are granted only the permissions necessary to perform their tasks, enhancing the security of organisation's AWS infrastructure.
\par
Overall, AWS IAM allows organisations to create tailored permissions for accessing only those AWS Glue resources that they want to analyze in Manta Flow, which builds their trust in this solution.
\par
It is necessary to generate programmatic user credentials to be able to access AWS services from the SDK. These credentials consist of an access key ID and a secret key. This has to be done by organisation's AWS administrator.

\subsection{ETL job metadata}
Let us have a look at what ETL job metadata is available in SDK so we can determine what we can use for data flow analysis. Below is a comprehensive list of interesting properties of job metadata that are relevant for data lineage analysis. We decided to omit some non-relevant ones as the complete list is too extensive~\cite{gluejobs}.
\begin{itemize}
    \item \textbf{Name} – string that identifies a job.
    \item \textbf{Command} – A \texttt{JobCommand} object containing detail about the executed command.
    \begin{itemize}
        \item \textbf{Name} – string identifying command type. There are 3 command types: \texttt{glueetl} is a standard Spark job, \texttt{pythonshell} is a job using standard Python shell (without Spark), \texttt{gluestreaming} is a streaming Spark job that runs continuously consuming data from streaming sources such as Apache Kafka. 
        \item \textbf{ScriptLocation} – string specifying the Amazon S3 path to a script that runs a job.
        \item \textbf{PythonVersion} – string specifying the Python version being used to run a Python shell job. Python scanner only supports Python version 3.    
    \end{itemize}
    \item \textbf{DefaultArguments} – A map array of key-value pairs where each key and value are strings. Contains the default arguments for every run of this job.
    \item \textbf{NonOverridableArguments} – A map array of key-value pairs. Contains arguments that cannot be overriden.
    \item \textbf{Connections} – A list of connections used for this job.
    \item \textbf{GlueVersion} – string specifying AWS Glue version which determines the versions of Apache Spark and Python available in a job.
\end{itemize}
\par
Each ETL job has a unique name that can be used for its identification. A job executes a specific command that consists of the environment, job script and arguments.
\par
There are 3 different command types but none of them influences how Python scanner analyzes the script. Presence or absence of Spark environment can be inferred from the script. When Spark functions are used, we can expect that Spark is available, otherwise the script would fail.
\par
Each script is stored in Amazon S3 which means that we need to download it from there in order to analyze it. Users are required to provide sufficient permissions for this action when configuring credentials.
\par
Each ETL job can be parameterized by arguments which consist of default, non-overridable and standard arguments defined on a job run. A complete set of arguments used to run a job can only be found by examining the history of job runs. Arguments can be accessed in the script by calling a dedicated function. Some of the arguments can be used to set up the script environment for the jobs and job runs. There are 4 of them which we need to be aware of:
\begin{enumerate}
    \item \texttt{-{}-additional-python-modules} specifies a list representing a set of Python packages to be installed. It is possible to install packages from PyPI (Python Package Index) or provided in a custom distribution. A custom distribution entry is the Amazon S3 path to the distribution. These packages are available to be used in the job script.
    \item \texttt{-{}-extra-files} contains a list of Amazon S3 paths to additional files, such as configuration files that AWS Glue copies to the working directory of the script before running it. These files can be referenced in the script using a relative path.
    \item \texttt{-{}-extra-py-files} contains a list of Amazon S3 paths to additional Python modules that AWS Glue adds to the Python path before running the script. These modules are available to be used in the job script.
    \item \texttt{-{}-scriptLocation} contains an Amazon S3 location where the ETL script is located. This parameter overrides the script location set in job metadata.
\end{enumerate}
\par
Lastly, jobs can use certain \textit{Connections} defined in Data Catalog. Only the Connections specified in this job parameter can be used. We will explain what they are in the following section.

\subsection{Data Catalog metadata}
We shall also look at available metadata for Data Catalog. There are three types of entities that are useful for data lineage analysis: databases, tables and connections. Firstly, let us list important properties of database metadata~\cite{gluedatabase}:
\begin{itemize}
    \item \textbf{Name} - string containing the name of the database.
    \item \textbf{CatalogId} - the ID of the Data Catalog in which the database resides.
\end{itemize}
The following list contains important properties of table metadata~\cite{gluetable}:
\begin{itemize}
    \item \textbf{Name} - string containing the table name.
    \item \textbf{DatabaseName}  - string specifying the name of the database where the table metadata resides.
    \item \textbf{StorageDescriptor} - \texttt{StorageDescriptor} object describing the physical storage of table data.
    \begin{itemize}
        \item \textbf{Columns} – An array specifying columns of the table.
        \item \textbf{Location} – Location string containing URI of the physical location of the table.
    \end{itemize}
    \item CatalogId - the ID of the Data Catalog in which the table resides 
\end{itemize}
Finally, a list enumerating important properties of connection metadata:
\begin{itemize}
    \item \textbf{Name} – string containing the name of the connection definition.
    \item \textbf{ConnectionType} – string specifying the connection type, one of \texttt{JDBC}, \texttt{SFTP}, \texttt{MONGODB}, \texttt{KAFKA}, \texttt{NETWORK}, \texttt{MARKETPLACE}, \texttt{CUSTOM}.
    \item \textbf{ConnectionProperties} - a map array of string key-value pairs. These key-value pairs define various parameters of the connection, e.g. \texttt{HOST}, \texttt{JDBC\_CONNECTION\_URL} etc.
\end{itemize}
\par
Data Catalog itself is identified by Catalog ID, which is the same identifier as the 12-digit ID of the AWS account to which it belongs. AWS account could be understood as organisation's account under which all AWS resources are grouped, although it is possible for organisations to have multiple accounts. In general, an AWS account can use one AWS Glue service instance in each AWS region (geographical regions specifying data centers) and each service instance has a single Data Catalog. It is not possible to use Data Catalogs in different regions, but it is possible to use Data Catalogs belonging to different AWS accounts. A Data Catalog is therefore uniquely identified by Catalog ID (AWS account ID) and AWS region.
\par
A Catalog contains databases which represent a logical grouping of tables. Since Data Catalog is a metadata repository, it directly does not contain any data, it just contains metadata about data sources. Data Catalog databases are just containers containing any arbitrary grouping of tables which may describe resources stored in different locations.
\par
Tables represent a collection of related data organized in columns and rows. Each table maps to a data source for which it provides connection details, so it is possible to trace the real data location, which is crucial to provide complete data lineage. Such data source may be a relational database table, a structured file or any other data source for which there is a connector available. The benefit of a Catalog table is that no connection details have to be provided when such table is used in an ETL job or other AWS service and it also contains schema information, which is especially helpful for resources that do not directly provide it (e.g. files). 
\par
Connections can be used to store connection details for commonly used data sources such as databases. They allow secure storage of connection credentials so they do not need to be specified in plain-text in code. This connection also becomes a single source of truth for connection details, so if its settings change, they only need to be changed in one place. A connection is specified by its type, which defines the connector that AWS Glue will use to load and save data, and a collection of connection properties. Analyzing Connections is an important metadata information for data lineage analysis, because when they are used in code, the connection details are unknown and have to be provided externally.

\subsection{Analyzing ETL jobs}
We now have enough information to think about how we can analyze data lineage in ETL jobs. The process looks rather simple. ETL jobs consist primarily of the job script, which we can analyze using Embedded Code Service. Additional metadata about the execution environment can be passed in the configuration argument of the service. This configuration shall contain argument values as well as certain Data Catalog metadata which are necessary to successfully recognize data sources used in the script. AWS Glue does not process any data outside of job scripts so at this point we will not need to map any pin nodes.

\subsubsection{Python scanner improvements}
\label{sec:python}
There are certain Python scanner improvements required to support analyzing AWS Glue scripts. The scanner is already capable of analyzing Spark code as it supports the analysis of \textit{PySpark} library (Python API for Apache Spark) function calls, but AWS Glue introduces an extension of this library called \textit{awsglue}. This library provides additional functionality for working with Spark in AWS Glue environment and for using other features of AWS Glue such as Data Catalog. Figure~\ref{fig:visualScript} contains several function calls from this library. The scanner needs to be extended with a plugin for handling these function calls.
\par
The main feature of this library is the \texttt{DynamicFrame} class, which is similar to PySpark's \texttt{DataFrame}. It can be created from AWS Glue Data Catalog table, connection or some other data source available in AWS and it can be converted from and to a \texttt{DataFrame}. It supports common operations over data frames such as filtering, mapping, joining, etc. There are also custom ETL transformations defined for these frames, so the users do not need to convert to \texttt{DataFrame} for common use-cases. These transformations are used when the code is generated from graphical tool in AWS Glue.
\par
The AWS Glue \texttt{getResolvedOptions(args, options)} utility function gives access to the arguments that are passed to the script when a job is ran. Job arguments are a useful tool that makes a job dynamic and modular and are therefore used quite often. We do not have direct access to the contents of the \texttt{args} argument (that is usually supplied from \texttt{sys.argv} - command line arguments), but using Outsight we can supply it from AWS Glue scanner. Job metadata contain default arguments, it is also possible to analyze job run history and collect the different values with which the job was executed, or simply allow users to provide these values in a handy format. With this Outsight, we can simply construct a dictionary in the collaborative propagation mode containing the keys defined in the \texttt{options} argument, which is usually a list containing string constants. Figure~\ref{fig:resolvedOptions} demonstrates the usage of this function.
\begin{lstlisting}[language=Python,caption=Usage of \texttt{getResolvedOptions} function,label=fig:resolvedOptions]
import sys
from awsglue.utils import getResolvedOptions

args = getResolvedOptions(sys.argv, ['JOB_NAME', 'day_partition_key', 'hour_partition_key', 'day_partition_value', 'hour_partition_value'])

print "The day-partition key is: ", args['day_partition_key']
print "and the day-partition value is: ", args['day_partition_value']
\end{lstlisting}
\par
\texttt{GlueContext} object wraps the Apache Spark \texttt{SparkContext} object, and thereby provides a mechanisms for interacting with the Apache Spark platform. It contains functions for creating data sources and data frames, working with datasets in Amazon S3, managing transactions and writing data. We are mostly interested in the functions that create and write data frames as they provide inputs and outputs of the ETL jobs. \texttt{GlueContext} can create \texttt{DynamicFrames} or \texttt{DataFrames} from Data Catalog or from options. The Data Catalog way is quite simple as only the database and table name is provided. These values are enough to match the table in the Data Catalog. Creating a frame from options is a bit trickier as there are multiple types of available connections and each of them consumes different options. However, these options are string values for which we already have a handful of mitigation strategies. Writing methods allow writing data frames in a similar way as reading them. It is possible to write both \texttt{DynamicFrames} and \texttt{DataFrames} to Data Catalog by providing name of the database and table. In advance to that, \texttt{DynamicFrame} can also be written from options or using a stored JDBC connection.
\par
Transformation classes are a different approach to apply transformations to \texttt{DynamicFrames}. Internally, they call the relevant function on the frame. A difficult problem to solve is how the transformation is applied. Each transformation is inheriting from parent class \texttt{GlueTransform}, which declares a couple of class methods. Most of them are not interesting and only provide information to user about themselves. The interesting one is the \texttt{apply(cls, *args, **kwargs)} method, which applies the transformation with given arguments. This method is not overridden in child classes. Basically what it does internally is creating an object of the inheriting class type using the \texttt{cls} parameter. As it is a class method, this parameter is auto-filled by Python and contains the reference to the current class. Then, this object is invoked with the remaining arbitrary arguments, which in fact means that the\texttt{\_\_call\_\_} method of the child class is invoked. This is where the transformation method is invoked on the data frame object with the right arguments. This creates a tricky situation where each transformation uses the same \texttt{apply} method, so the current algorithm for invocation target resolution cannot reliably solve this issue. An additional improvement could be to resolve invocations based on the calling object, and if that object turns out to be a class, we can only look for the methods of that class. This solution is a mitigation strategy for an imperfect algorithm for invocation target resolution, but can be implemented rather easily.

\subsection{Analyzing Data Catalog}
It is obvious that since Data Catalog tables are used as data sources and sinks in ETL jobs, they are an integral part of the data flow. Less obvious is the fact that mapping from a table to data location (\texttt{Location} is the name of the metadata attribute containing URI to data location) also needs to be visualized. Take for example a situation where one process produces data and stores it on Amazon S3 in a file called \texttt{foo.csv} and another pipeline reads the data from Data Catalog table mapped to the S3 location of \texttt{foo.csv}, applies transformations and stores the result to a different Data Catalog table. Without the link between the data location and Data Catalog table, the graph would be disjointed.
\par
It is also important to review whether the nodes for Data Catalog tables should exist in graph. Since the tables themselves only represent a different data source, we could replace the table node with the node of the data sources directly in the graph. While such representation would be technically correct, it would hide the semantics of using a Data Catalog table. Those users that are aware of the usage of Data Catalog tables but are unaware of the underlying data sources will not understand the graphs correctly. There may also be scenarios where only the Data Catalog tables are used in data pipelines and the underlying data source is never referenced directly. In such scenarios replacing the table node would be confusing. As we can’t confirm that such situations would not occur, we cannot hide this information in the graph, therefore both nodes and the edge between them need to be created. An example of how such data lineage could be visualized is shown in Figure~\ref{fig:catalogLineage}. This example is based on the code shown in Figure~\ref{fig:visualScript}. Blue nodes represent actual files, green nodes represent Data Catalog tables for these files, yellow nodes are a part of Python data lineage.

\begin{figure}[ht]\centering
\includegraphics[width=1.0\textwidth]{img/catalog_lineage.png}
\caption{An example of data lineage graph containing Data Catalog tables}
\label{fig:catalogLineage}
\end{figure}

\section{Design of AWS Glue scanner}

The analysis from the previous section creates a strong foundation upon which we can build the design of AWS Glue scanner. We were thinking ahead when creating this design so it covers not only the features implemented in the MVP included in this work, but also the features that while not required for the purposes of the MVP, need to be implemented in near future to create a full-fledged scanner.
\par
AWS Glue scanner is designed in a standard way consisting of two main components: Connector and Dataflow Generator. The task of the Connector is to connect to AWS Glue, extract all required metadata and transform it into a general model that can be used for data flow analysis. Dataflow Generator uses this model to analyze data flows and create a data lineage graph. 

\subsection{Connection scope}
Each scenario executed in Manta Flow CLI analyzes a single connection to a data technology. The first thing we need to specify is the scope of a connection in AWS Glue, that is, what entities are analyzed in one scenario execution. Usually, this scope is defined by connection URL (where available) and credentials. A scenario then analyzes everything it has access to using these connection settings. It makes sense to use this approach in AWS Glue as well. A connection in AWS Glue scanner is defined by credentials and the AWS region. This pairing uniquely identifies the AWS Glue service instance (a service instance can be identified by an AWS account ID, which we can obtain from credentials, and by an AWS region). This connection provides access to a set of ETL jobs and Data Catalog, which together represent the connection's scope of the data flow analysis. The access can be restricted by permissions set in AWS IAM service.

\subsection{AWS Glue Connector design}
AWS Glue Connector takes care of extracting and storing metadata from AWS Glue and resolving the inputs for data flow analysis. The connector is divided into 4 main components, as is common with other scanners:
\begin{enumerate}
    \item \textit{Extractor} which extracts metadata from AWS Glue
    \item \textit{Model} which contains the definition of the general model of the input
    \item \textit{Resolver} which is an implementation of the general model
    \item \textit{Reader} which reads extracted metadata into the model used by Dataflow Generator
\end{enumerate}

\subsubsection{Extractor}
The extractor connects to the AWS Glue service instance and extracts all required metadata, saving them on the file system. To extract the metadata, AWS Java v2 SDK is used. The SDK returns data in its own Java objects. We need to extract Data Catalog metadata and ETL job metadata including job scripts. We must not forget that we also need to extract additional libraries and files which are specified in job arguments.
\par
Extracted metadata must be stored in a convenient format and in a well-defined hierarchy on the file system so that it can be correctly loaded into the model used in the data flow analysis. It is common to store metadata in the format in which it was extracted from data technology. In a situation where retrieval of a particular artifact fails (due to insufficient rights or other error), the user can provide it manually. Users can export AWS Glue metadata in multiple formats (using \texttt{aws-cli}, a command line tool for interacting with AWS services), so we chose JSON format for convenience.
\par
The file hierarchy for extracted metadata has to be unambiguous so that the Reader can read it correctly. We designed the file hierarchy shown in Figure~\ref{fig:hierarchy}. Names enclosed in \texttt{< >} represent variable names based on the actual name of the entity described between \texttt{< >} (\texttt{<job1>} would be replaced by the actual name of the first extracted ETL job etc.). The hierarchy intentionally uses \texttt{<region>} and \texttt{<account-id>} top-most directories. While a connection can only extract metadata in a single region, the name of the region is not a part of any metadata, but is required for correct naming of resources, so we store this value in the name of the directory. Account ID can be inferred from Catalog ID stored in table and database metadata, but this value is not present in job metadata and we need it to correctly resolve which Data Catalog is used in job scripts (if no Catalog ID is used when accessing Data Catalog resources, the default one is used), so we stored it in the directory name as well. However, it has another reason. It is possible to use Data Catalog belonging to another AWS account in ETL jobs, so when its metadata is extracted, it is stored in a different directory under that account's ID. Then there are two directories, \texttt{jobs} directory containing ETL job metadata and \texttt{data\_catalog} directory containing Data Catalog metadata. ETL job metadata consist of metadata JSON file and script file, if the job uses additional libraries or files, they would also be stored here. Data Catalog metadata contain JSON files of databases and tables.

\begin{figure}[ht]\centering
\includegraphics[width=1.0\textwidth]{img/file_hierarchy.png}
\caption{File hierarchy of extracted AWS Glue metadata}
\label{fig:hierarchy}
\end{figure}

\subsubsection{Model, Resolver and Reader}
Model component defines a common data interface of the entities of the general model of AWS Glue input following the principle of loose coupling.
\par
Resolver component contains implementations of Model interfaces. Classes are designed to be immutable so the input for the data flow analysis cannot be accidentally modified.
\par
Reader component takes care of reading the extracted metadata and creating its object representation using the classes defined in Resolver.

\subsection{AWS Glue Dataflow Generator design}

AWS Glue Dataflow Generator analyzes data flows in the extracted inputs and creates a data lineage graph. Analyzing ETL jobs is fairly simple, all that the Generator needs to do is to call Embedded Code Service. After that it merges the result graph into the AWS Glue graph containing a node representing the job and the work is done. A more interesting problem is the analysis of Data Catalog metadata.
\par
The goal of Data Catalog analysis is to create data flow edges between the nodes representing Data Catalog tables and the nodes representing the actual data sources as well as adding edges when these tables are used in ETL jobs. There are two possibilities how we can achieve that.
\par
The first approach can create a more precise data lineage, but this lineage is only created when Data Catalog is used from AWS Glue (other AWS services can also use Data Catalog, for example AWS Athena can use Data Catalog tables in SQL queries). When ETL jobs are analyzed by Embedded Code Service, the resulting graph shall contain pin nodes representing reads and writes to Data Catalog tables. Then, Dataflow Generator would create the node for this table and map the pin node to the table node. Dataflow generator would also create data source node that the table is mapped to and link it with the table node. In case of Python, table schema could be passed directly to Python analysis using Outsight to provide a more detailed column-level lineage.
\par
The second approach is a more general solution. Firstly, Data Catalog metadata would be stored in a data dictionary so it could be accessed by Dataflow Query Service. Since mapping between the data source and Data Catalog table is visualized as a data flow, it is also possible to create these flows in an extra data flow scenario. Such scenario would create data flows from data sources to Data Catalog tables for all tables present in extracted metadata. Python scanner would use Dataflow Query Service to resolve Data Catalog accesses without the need to create any extra data flows to data sources, because they would already exist. However, since there would be only dataflows from data sources to catalog tables, edges in the other direction (backlinks, when data is written to Data Catalog table) would have to be added in a postprocessing scenario.
\par
The second solution provides more value and the lineage can also be reused for other scanners, that is why we prefer it. However, it implies that several new components need to be developed, namely:
\begin{enumerate}
    \item \textit{Data dictionary mapping scenario} for mapping Data Catalog metadata into a data dictionary
    \item \textit{Data Catalog data flow scenario} for creating data flows between Data Catalog tables and their data sources
    \item Specific Dataflow Query Service implementation for AWS Glue Data Catalog
    \item Backlink mapping configuration for adding a missing edge between Data Catalog table and data source when data is written to the table
\end{enumerate}

\section{Implementation of AWS Glue scanner}
In this work we implemented the MVP of AWS Glue scanner. The goal of this MVP was not to create a full-fledged scanner, but to be able to demonstrate the functionality of Embedded Code Service. The prototype can be extended in the future following the presented design. As some of the designed features are implemented in the MVP and some are not, here is a comprehensive list that sums it up:
\begin{itemize}
    \item Implemented features
    \begin{itemize}
        \item Extraction of ETL job metadata and scripts
        \item Data flow analysis of ETL jobs
        \item Creating data lineage graph
        \item Integration with Manta Flow
        \item Configuration in Admin UI
        \item Plugin for analyzing \textit{awsglue} library function calls in Python scanner
        \item Agent for AWS Glue
    \end{itemize}
    \item Unimplemented features
    \begin{itemize}
        \item Extraction of Data Catalog metadata
        \item Extraction of additional files
        \item Data flow analysis of Data Catalog    
    \end{itemize}
\end{itemize}
\par
Let us mention interesting parts from the implementation of some of the features.

\subsection{Extraction}
AWS SDK used for metadata extraction always provides responses to requests deserialized in the form of a Java object. We have stated that we want to store metadata in a serialized JSON format. To avoid implementing serialization logic, we developed a response \textit{interceptor} (the \texttt{GlueClientExecutionInterceptor} class) that intercepts a HTTP response before it is deserialized. At this point, the body of the response contains the metadata in the desired JSON format, so we copy it and let the response be deserialized, because it is also convenient to read some of the metadata from the provided response object.
\par
We have also implemented a Manta Flow Agent specialization for AWS Glue extraction. Manta Flow Agent is an application for metadata extraction. In enterprises, it is common to limit access to certain networks for security reasons. Agent was created to allow extracting metadata from systems that are not accessible from the same network as Manta Flow Server.

\subsection{Manta Flow integration}
AWS Glue scanner is fully integrated with Manta Flow. It is released as one of so-called preview scanners which can be used only in preview mode. The scanner can be configured using Admin UI as all other scanners. The configuration includes specification of credentials and filtering expressions for ETL job names that should be included in the analysis.

\subsection{Plugin for awsglue library}
While not directly a part of AWS Glue scanner, we have developed a plugin for Python scanner that analyzes function calls in \texttt{awsglue} library. This plugin was important to be able to analyze Python code used in AWS Glue as the function calls are often present in it. We have conducted the analysis of this library in Section~\ref{sec:python}, which sums up the range of propagation modes that need to be implemented. We have implemented the functional base of the plugin that handles the extraction of this library and invocation of its propagation modes. We have also developed two propagation modes.
\par
The first propagation mode handles \texttt{create\_dynamic\_frame\_from\_options} function. This function is used to read external data specified by the options into a \texttt{DynamicFrame}. The options are key-value pairs of string values and define the connection details of the data source. The propagation of this function is implemented in  \texttt{CreateDynamicFrameFromOptionsPropagationMode}. The propagation mode works by trying to resolve the connection options in the best-effort way. We can split the options into two sets: stream connection options and database connection options. The propagation mode creates data read flow when it matches at least a part of the connection details and provides placeholder values for missing properties. If no option can be resolved, no flow is created. That is because we have no information about whether a stream or a database is accessed. This flow is wrapped in a flow representing an unknown column of a \texttt{DynamicFrame} and propagated to the target, which is the return value of the function.
\par
The other propagation mode handles the \texttt{toDF} function of \texttt{DynamicFrame} class that converts this frame into PySpark \texttt{DataFrame}. Developers often prefer PySpark frames to AWS Glue frames, because they are used to them and are more capable. First, the propagation mode \texttt{DynamicFrameToDataframePropagationMode} discovers all flows representing a column in a \texttt{DynamicFrame} and transforms them to a flow representing a \texttt{DataFrame} column, preserving column name and the source of the data in the column. Transformed flows are propagated to the target, which is the return value of the function. These flows can then be used by the plugin that handles function calls of PySpark library to resolve any function calls on the returned data frame.

 

 
\chapter{Evaluation}

In this chapter, we evaluate the outcomes of the implementation presented in the preceding chapters. The primary goal of our work was data flow analysis of embedded code in the context of Manta Flow and we need to assess whether our proposed solution achieved its intended objectives.
\par
In this evaluation, we will use an example of an AWS Glue ETL job. This example demonstrates the usage of Python Embedded Code Service for data flow analysis of embedded code in AWS Glue. It makes sense to evaluate both of these components together, because that is how they are intended to be used. We will present and explain the source of the example and then we will show and assess the resulting data lineage with respect to the predefined objectives.
\par
Note that the presented example was created to showcase the implemented functionality and might not be a real representation of an AWS Glue job. However, it does not mean that the data flow makes no sense or that actual scripts would be completely different, but rather that they would be structured differently and contain additional logic for logging etc. The example is also limited by unimplemented features because some function calls that would be used in a production code are not yet supported in Python scanner.

\section{ETL job example}

The example that we are going to use for evaluation demonstrates a simple ETL job for data transformation using AWS Glue. The script code is shown in Figure~\ref{fig:evaluationCode}. It showcases several features implemented in this work as well as some common features of Manta Flow to show that the implemented solution is well-integrated. At the same time, we tried to keep the example reasonable so it is similar to how ETL jobs are usually implemented.
\par
The example is a simple ETL job defined without job arguments. It uses AWS Glue to facilitate data reading, but it does not use Data Catalog. The pipeline that we created reads data from an Amazon Redshift view into a \texttt{DynamicFrame}. This frame is converted to PySpark \texttt{DataFrame} for convenience. Next, we apply a simple data transformation to the \texttt{DataFrame}. We just added a new column, the exact transformation does not matter because its details would not be shown in the graph anyway. Python scanner does not show details about internal transformations. The transformed \texttt{DataFrame} is stored on the local file system in the CSV format. Finally, the CSV file is uploaded to Amazon S3. In more detail:
\begin{itemize}
    \item Lines 1--4 contain library imports. In our example we need \texttt{awsglue} and \texttt{pyspark} libraries which we have already introduced and \texttt{boto3} library which is a Python library for working with AWS services.
    \item Line 6 defines a JDBC URL for the Redshift database that we use.
    \item On lines 8--18 we define the \texttt{get\_df} function that uses AWS Glue context object to read data from the Redshift database. It is common to use this context to read or write data, because AWS Glue can safely facilitate this data connection using a Redshift connector. The returned data frame is converted from AWS Glue \texttt{DynamicFrame} to PySpark \texttt{DataFrame}. This conversion is also used often as developers are more familiar with PySpark library rather than AWS Glue and it is also far more capable.
    \item On lines 20--21 we define data transforming function \texttt{transform\_df}. As we already mentioned, the actual transformation is not important for this example, because it does not showcase any feature implemented in this work. The function is supposed to represent any set of data frame transformations.
    \item Lines 23--32 are the main body of the script. Firstly, we initialize Spark and AWS Glue contexts. We then retrieve input data frame using the \texttt{get\_df} function and apply the transformations on it. On line 29 we store the data frame to a local CSV file using PySpark CSV writer. Because the job is executed in a serverless environment of AWS Glue, this file would be erased together with the job's working directory when the execution ends. To preserve the created file, we upload it to Amazon S3 on line 32. Although AWS Glue provides a more convenient way for working with S3 resources, many developers prefer using this common Spark approach. 
\end{itemize}

\begin{lstlisting}[language=Python,caption=Source code of an ETL job demonstrating embedded code analysis,label=fig:evaluationCode]
from awsglue.context import GlueContext
from pyspark import SparkContext
from pyspark.sql.functions import lit
import boto3

jdbc = "jdbc:redshift://dev.eu-central-1.redshift.amazonaws.com:1234/automated_test"

def get_df(jdbcurl,table):
    my_conn_options = {
        "url": jdbcurl,
        "dbtable": table,
        "user": "masterUsername",
        "password": "masterUserPassword",
        "redshiftTmpDir": "s3://testdir/testbucket",
        "aws_iam_role": "glue_execution_role"
    }
    df = glueContext.create_dynamic_frame.from_options(connection_type="redshift", connection_options=my_conn_options)
    return df.toDF()

def transform_df(df):
    return df.withColumn("PROFIT", lit(None))

if __name__ == "__main__":
    sc = SparkContext.getOrCreate()
    glueContext = GlueContext(sc)
    
    in_df = get_df(jdbc, "sales_view")
    csv_df = transform_df(in_df)
    csv_df.write.csv("sales_data.csv")
    
    s3_client = boto3.client("s3")
    s3_client.upload_file("sales_data.csv", "mysalesbucket", "sales_data.csv")
\end{lstlisting}


\par
The created data lineage graph can be seen in Figure~\ref{fig:thesisDemo1}. Figure~\ref{fig:thesisDemo2} shows the same graph zoomed in on its left side, Figure~\ref{fig:thesisDemo3} is zoomed in on the right side. Figure~\ref{fig:thesisDemo4} shows the left part of the graph when entire Redshift database is also visualized to demonstrate that the scanner provides data lineage integrated with other data technologies. The visualized object is the AWS Glue ETL job shown in black border. The purple border represents the job script. Yellow nodes represent Python data lineage, green nodes represent files stored on a file system and red nodes represent Redshift data lineage.
\par
We can see that the graph contains a complete data lineage as described by the example. Redshift data flows into the node representing data reading operation in Python code and into the node representing write operation. The data is stored in a file \texttt{sales\_data.csv} located on localhost from where it is uploaded to Amazon S3. The data flow operation of uploading the CSV file to S3 consists of two Python nodes, because the file is first read from the file system and then uploaded to S3.
\par
We can see that the data lineage does not contain the exact information about propagated columns. It is due to the implementation of Python scanner that currently cannot use schema information during data flow analysis. It works with a concept of unknown columns in propagations and external data source nodes are only added in Dataflow Generator.
\par
Lastly, we would like to present performance data from data flow analysis. This data was not measured using any precise profiling technique, but rather collected by observing the logs that were produced during the execution. There are 4 interesting timestamps in the log:
\begin{enumerate}
    \item \textbf{16:44:16.847} - AWS Glue data flow scenario started
    \item \textbf{16:44:16.959} - Python scanner started analyzing embedded code
    \item \textbf{16:45:02.380} - Python scanner finished analyzing embedded code
    \item \textbf{16:45:02.938} - AWS Glue data flow scenario ended
\end{enumerate}
We can observe that it took a little over 0.1s to initialize the AWS Glue scenario and perform input orchestration in Embedded Code Service. Python analysis took approximately 45.4s. After that, in less than 0.6s the graphs were merged, at which point the execution of Embedded Code service has finished, and the finalization steps of the scenario were executed. In conclusion, the execution of AWS Glue scanner and Embedded Code Service code took around 0.7s cummulatively (1.5\% of the overall time) while the execution of Python scanner took around 45.4s (98.5\% of the overall time). These results are not very precise, but we can see that Embedded Code Service is sufficiently optimised and does not create a major performance bottleneck. The effort to speed up the overall analysis should be focused mainly on Python data flow analysis.


\begin{figure}[ht]\centering
\includegraphics[angle=90,origin=c,height=1.0\textwidth]{img/thesis_demo1.PNG}
\caption{Data lineage graph created by analyzing the demonstration script}
\label{fig:thesisDemo1}
\end{figure}

\begin{figure}[ht]\centering
\includegraphics[angle=90,origin=c,height=1.0\textwidth]{img/thesis_demo2.PNG}
\caption{Left part of the data lineage graph}
\label{fig:thesisDemo2}
\end{figure}

\begin{figure}[ht]\centering
\includegraphics[angle=90,origin=c,height=1.0\textwidth]{img/thesis_demo3.PNG}
\caption{Right part of the data lineage graph}
\label{fig:thesisDemo3}
\end{figure}

\begin{figure}[ht]\centering
\includegraphics[angle=90,origin=c,height=1.0\textwidth]{img/thesis_demo4.PNG}
\caption{Right-most node of the data lineage graph together with a part of Redshift data lineage}
\label{fig:thesisDemo4}
\end{figure}

\section{Limitations and Future Work}

As demonstrated on the example, Python Embedded Code Service designed and implemented in this work can be integrated with other data technology scanners and is capable of analyzing embedded Python code.

\subsection{Other programming languages}
We have only been able to implement the service for embedded Python code. The analysis of C\# and Java code is not yet supported by any Embedded Code Service. While the implementation was not set as our goal, we have provided a comprehensive guide how these services can be developed should they be required.

\subsection{AWS Glue scanner}
We have developed a prototype of AWS Glue scanner that can analyze data lineage in basic ETL jobs. The range of developed features was sufficient to demonstrate the capabilities of Embedded Code Service, but is not yet sufficient to analyze any inputs provided by customers. 
\par
We have designed several more features that will need to be implemented in the future in order to cover basic capabilities of AWS Glue. Data Catalog analysis is a promising feature to unlock data lineage discovery in AWS Glue. The plugin for \texttt{awsglue} Python library has to be significantly extended to cover more commonly used functions and methods. The further development of the scanner will be a subject to prioritization based on customer preferences.

\subsection{Python scanner improvements}
We have implemented only a few improvements of Python scanner. We needed to redesign several components so the scanner can be integrated with Embedded Code Service and we developed an \texttt{awsglue} library plugin. These changed allowed us to analyze the example script used in this evaluation.
\par
To provide a better value, Python scanner will need to be optimized in the future for the analysis of embedded code. We have observed that embedded Python code is often parameterized, e.g. using job arguments in AWS Glue. Python scanner currently provides poor interface for processing external values and is not optimized to analyze large sets of parameters.



\chapter{Conclusion}

In this work, we managed to successfully develop Python Embedded Code Service for data flow analysis of embedded Python code as well as a prototype of AWS Glue scanner, which uses this service to analyze ETL job scripts. The scanner is fully integrated in Manta Flow production deployment.
\par
The service currently supports only AWS Glue, but can easily be extended to support other data technologies by implementing specific orchestration process for that technology. The analysis of embedded code is currently limited to Python code, but we have explained clear steps and necessary changes that need to be made to scanners and the service in order to support a different programming language in the future.
\par
AWS Glue scanner prototype is able to extract and analyze data flow lineage in ETL jobs. We have conducted the analysis and designed how the scanner can be extended to also support analyzing data flows in Data Catalog.
\par
We have extended Python scanner with a basic implementation of the plugin for analyzing function calls in \texttt{awsglue} library often used in AWS Glue ETL jobs. The plugin is ready for development of new propagation modes for function commonly used in ETL jobs.
\par
In the last chapter we have shown that AWS Glue scanner is able to successfully analyze embedded code using Python Embedded Code Service to create a data lineage graph for AWS Glue, internally employing Python scanner for data flow analysis.
\par
The future development should focus on finishing the unimplemented features in AWS Glue scanner to make it a full-fledged scanner and to optimize Python scanner along with Embedded Code Service to be more effective in analysis of embedded code.


%%% Bibliography
%\include{bibliography}
\printbibliography

%%% Figures used in the thesis (consider if this is needed)
%\listoffigures

%%% Tables used in the thesis (consider if this is needed)
%%% In mathematical theses, it could be better to move the list of tables to the beginning of the thesis.
%\listoftables

%%% Abbreviations used in the thesis, if any, including their explanation
%%% In mathematical theses, it could be better to move the list of abbreviations to the beginning of the thesis.
%\chapwithtoc{List of Abbreviations}

%%% Attachments to the master thesis, if any. Each attachment must be
%%% referred to at least once from the text of the thesis. Attachments
%%% are numbered.
%%%
%%% The printed version should preferably contain attachments, which can be
%%% read (additional tables and charts, supplementary text, examples of
%%% program output, etc.). The electronic version is more suited for attachments
%%% which will likely be used in an electronic form rather than read (program
%%% source code, data files, interactive charts, etc.). Electronic attachments
%%% should be uploaded to SIS and optionally also included in the thesis on a~CD/DVD.
%%% Allowed file formats are specified in provision of the rector no. 72/2017.
\appendix
\chapter{Attachments}

\section{First Attachment}

\openright
\end{document}
